\documentclass[11pt, a5paper, twoside, chapterprefix, openany]{scrbook}
\usepackage[utf8x]{inputenc}
\usepackage{scrlayer-scrpage}
\usepackage[all]{nowidow}
\usepackage[kerning, spacing]{microtype}
\microtypecontext{spacing=nonfrench}
\usepackage[english]{babel}

\usepackage[pdfauthor={Rabbi Benzion Meir Hai Uzziel},
  pdftitle={Hegionei Uzziel},
  pdfsubject={Jewish thought},
  pdfproducer={Simon Montagu, Hataf Segol Publishing},
  pdfcreator=pdflatex,
  colorlinks=false, linkcolor=blue,
  linkbordercolor={0 0 1}]{hyperref}

\makeatletter
\Hy@AtBeginDocument{%
  % Override border definition set with colorlinks=true
  \def\@pdfborder{0 0 1}
  % Overrides border style set with colorlinks=true
  % Hyperlink border style will be underline of width 1pt
  \def\@pdfborderstyle{/S/U/W 1}
}
\makeatother

% Page styles\
\parskip 8pt
\parindent 0pt
\usepackage{indentfirst}
\parindent 8pt

\usepackage{times}

% Use the text font for chapter headings and titles
\pagestyle{scrheadings}
\addtokomafont{title}{\normalfont}
\setkomafont{chapterprefix}{\normalfont\large}
\setkomafont{chapter}{\normalfont\Large}
\addtokomafont{chapterentry}{\normalfont}
\addtokomafont{partprefix}{\normalfont\bfseries\large}
\addtokomafont{part}{\normalfont\bfseries\Large}
\addtokomafont{partentry}{\normalfont}

\rohead{Hegionei Uzziel}
\lehead{Rabbi Benzion Meir Uzziel}
\ofoot{\thepage}

% Center chapter headings
\renewcommand*{\raggedchapter}{\centering}

\setcounter{secnumdepth}{-1}

\begin{document}

%half title page
\extratitle{
  \vspace*{2\baselineskip}
  \begin{center}Hegionei Uzziel\end{center}
    }
\title{Hegionei Uzziel}
\subtitle{Book 1}
\author{Rabbi Ben Zion Hai Meir Uzziel \and translated by Simon Montagu}
\date{Jerusalem, 1953 \\ English version 2025}

\maketitle

\frontmatter
\tableofcontents

\mainmatter
\part{God's Rewards}
\chapter{The Creation of the World}

Praised be the Creator and exalted be the Maker, who created his world with wisdom and made humanity, the crown of creation, in his image, as it is written, ``בראשית ברא אלהים'' — ``In the beginning God created…''\footnote{Genesis 1:1.} The Aramaic translation known as the Targum Yerushalmi translates this verse, “With wisdom God created,” and the message implied between the lines of this translation is stated explicitly in the holy Zohar: “Rabbi Yehudai said, what does ‘בראשית’ mean? It means the wisdom on which the world stands, the path leading into sublime and secret mysteries”.\footnote{Introduction to the Zohar 3b.} “What does ‘בראשית’ mean’? It means wisdom, the point within, concerning which the Psalm says, ‘All glorious is the king's daughter within.’\footnote{Psalm 45:14.} It means the awe which is the first commandment, concerning which it is written in the book of Proverbs, ‘The awe of the Lord is the beginning of knowledge.’\footnote{Proverbs 1:74}''\footnote{Tikkunei Hazohar, Tikun 30.} King David wrote similarly in the Psalms: “O Lord, how manifold are thy works! In wisdom you have made them all.”\footnote{Psalm 104:24.}

Knowledge of this wisdom, in so far as a human being can achieve it, is the source of love and awe for He who created the world and sustains it, as Maimonides writes in his halachic code, “When a person considers these things, and is aware of all created beings from the angels and the heavenly spheres to human beings and others, and perceives the wisdom of the Holy One, Blessed be He, in all the beings he created and formed, his love for God will increase; his soul will thirst and his body will yearn to love God, Blessed be He. He will be in fear and in awe due to his lowliness, his worthlessness, and his insignificance … he will see himself as a wretched and shameful object, empty and lacking.\footnote{Mishneh Torah, Hilchot Yesodei Hatorah, 2:2, 4:12.}

This is exactly the same idea that we quoted from the \textit{Tikkunei Zohar}: “This is wisdom … this is awe … concerning which it is written, ‘The fear of the Lord is the beginning of knowledge.’”

This is one of the foundations of the perfection of the world, and the wisest of all men, King Solomon,  summed it up in a few short words: “I know that, whatever God makes, shall last for ever; nothing can be added to it, nor any thing taken away from it; and God has made it, so that people should be in awe of Him.”\footnote{Ecclesiastes 3:14.} Rabbi Yehuda said, ``God created the world only so that he should be held in awe, as is is written, `God has made it, so that men should should be in awe of Him.’”\footnote{B. Shabbat 31a.} Maimonides interpreted this verse as referring to the eternity of the world and the renewal of its wonders: “He declares in these words that the world has been created by God and remains for ever. He gives a reason for its eternity by saying, ‘Nothing can be added to it, nor anything taken away from it.’ In other words, things change because they lack something or contain something superfluous, but since the works of God are totally perfect, with nothing needing to be added to them or removed from them, they must remain exactly as they are for ever. In the next part of the verse, Solomon, as it were describes the purpose of the things that exist, and provides a justification for changes to them, when he says, ‘And God has made it (that is, He performs miracles) so that people should be in awe of him.’”\footnote{Guide to the Perplexed 2:28, trsl. Friedlander.}

Maimonides means that when things happen outside the laws of nature, the reason
for it is to make people believe that they are God’s creation, and are like clay
in the hands of a potter who can do whatever he wishes with the world and
everything it contains, and so that they will be in awe of
Him.\footnote{Shem-Tov ibn Falaquera’s commentary ad loc.} This is all summed up
in Ribbi Yehuda's statement that “God only created His world,” i.e. the nature
of the world and the miraculous renewal in it are his world, which God created
in eternity and in wonderful variety, “so that the work of his hands should be
in awe of him.” From this we learn that the world, as a whole and in all its
details, was created through wisdom, and its objective is the perfection of
human beings, who were created with wisdom unlike every other creature, in order
to make them wise and intelligent with awe and love for God. This is the only
true wisdom, from which all other branches of knowledge spring like branches
from the trunk. They are all descendants and subdivisions of it, as the Rabbis
said, “God has nothing in his world but awe of Heaven, as it is written, ‘And
now, Israel, what does the Lord your God demand of you except to be in awe,' and
it is written, ‘And he said to Adam “Behold [\textit{Hen}]”’ — fear of God is
wisdom, since ‘\textit{Hen}’ is Greek for ‘one’”.\footnote{B. Shabbat 31b.}

\chapter{The Creation of the Human Race}

The creation of the human race comes at the end of the story of creation, since it unified all created beings. All visible and concealed forces in creation are dead bodies waiting for human action to discover them and activate them. What is more, human beings have a special power which even angels do not posess: the image of God on their face, as the verse says, “God created Man in his own image.” The rabbis commented on this verse, “Beloved is Man who was created in God’s image, but a special love was shown when it was revealed to him that he was Created in God’s image\footnote{Avot 3:14}.

In other words, the human race is more favoured than all creation since it was created in the divine image, and a special favour was made known to man when he was told that he was created in the divine image. As Maimonides writes, “Making known the extent to which we are favoured is a separate favour. Sometimes someone is granted a favour out of pity, and this is done in secret so as not to shame him,\footnote{Commentary to the Mishnah ad loc.} Rabbenu Yonah also writes, “Man would have been favoured by being created in the divine image even if it had not been made known to him that he was created in the divine image, but he was favoured after being created in God’s image when we were informed of this it was an extra favour\footnote{ibid.}

\chapter{The Image of God}

``And God created man in His own image, in the image of God created He him.'' This verse does not specify what ``the image of God'' means, and this has caused error among the ignorant, who lack the image of God and are unaware of His presence within themselves. They punctuate the verse incorrectly and read it, ``In the image, God created him'', as if God created man in the image. Ibn Ezra rejected this incorrect and misleading interpretation\footnote{Genesis 1:27}. Rashi also wrote ``God showed special favour by creating humans in his image ... and anyone who distorts the meaning by saying that he was created in the image is guilty of heresy.\footnote{Avot \textit{loc. cit}}

Ibn Ezra continues, ``[Sa`adia] Ga'on interprets `in our image, in our likeness' as `ruling,' by which he means in the image that He saw in wisdom as being good, and for the honour of the human race this was attributed to God.'' In other words, Sa`adia Gaon's interpretation of the verse is that man was created in the image that God saw as good, and ``in God's image'' means in the image of greatness of honour, as in the verses ``they went out of His land'',\footnote{Ezekiel 36:20} and ``The earth is the Lord's and the fulness thereof.''\footnote{Psalm 24:1}

This should never have been written: Ibn Ezra is putting a false interpretation into Sa`adia's mouth. Here is what Sa`adia actually wrote, translated from the Arabic: ``God created man in his image, He created him as a ruler.'' That is, the image of God in man is the image of rulership, just as the word \textit{elohim} in Hebrew means both ``God'' and ``judges,'' as in the verses ``the owner of the house shall approach \textit{elohim},''\footnote{Exodus 22:7} ``the cause of both parties shall come before \textit{elohim}; he whom \textit{elohim} condemn,''\footnote{\textit{ibid} 8} and ``Thou shalt not  curse \textit{elohim}.''\footnote{\textit{ibid} 27}. Sa`adia also translates ``The sons of \textit{elohim} saw the daughters of men''\footnote{Genesis 6:2} as ``the sons of the powerful,'' which is similar to Onkelos' Aramaic translation and Rashi's commentary ``The sons of the rulers and the judges.''

So when he writes ``He created him as a ruler,'' he means that ``in His image'' means that the image of man is the image of rule and domination, which is called ``the image of \textit{elohim}, and this is also the image of God as it is revealed to us in his omnipotent power. This is clear because Sa`adia did not write ``God created him as a ruler.''

\chapter{The Human Soul}
\chapter{Departure of God’s Presence}
\chapter{The Unique Living Soul}
\chapter{The Form of the Soul}
\chapter{The Soul of the Nation}
\chapter{Souls in Chaos}
\chapter{The Soul of Israel}
\chapter{The Interrelationship Between Humans and the World}
\chapter{Laws of Nature}
\chapter{The Principle of Reward and Punishment }
\chapter{Fear of God and Knowledge of God}
\chapter{The Definition of Reward}
\chapter{Rewards, Earnings and Prizes}
\chapter{The Rewards of Commandments}
\chapter{Commandments With Their Own Rewards}
\chapter{The Meaning of Reward}
\chapter{God’s Blessing and Punishment}
\chapter{Law and Justice}
\chapter{God’s Discipline}
\chapter{The Prayer of the Poor who Pours out his Complaint to God (Psalm 102).}
\part{Reward in general and in particular}
\chapter{Reward in general}
\chapter{Eternal reward}
\chapter{The right of reward}
\chapter{Law and justice in this world}
\chapter{Law and justice in the world of the soul}
\chapter{Eden and Gehinnam}
\chapter{The quality of justice in this world}
\chapter{The quality of justice in the next world}
\chapter{The quaility of mercy in justice}
\chapter{The next world}
\chapter{Who earns life in the next world}
\part{The nation and its rewards}
\chapter{The nature of the Torah personality}
\chapter{The flowering of Israel’s redemption}
\chapter{The father of the nation}
\chapter{The activities of the father of the nation in the land of Israel}
\chapter{Acquiring the land}
\chapter{The covenant of the land}
\chapter{The conquest of the land, its acquring and construction}
\chapter{The character of the nation}
\chapter{Justice and right}
\chapter{Defintions of right and justice}
\chapter{Legal Compromise}
\chapter{Right and justice in metaphorical sense}
\chapter{Justice}
\chapter{Zeal for justice}
\chapter{Sanctifying God’s name}
\chapter{Tribulation}
\chapter{The binding of Isaac}
\chapter{The purpose of the binding of Isaac}
\chapter{The sanctity of Jerusalem}
\chapter{Exile and redemption}
\chapter{Jacob’s dream}
\chapter{Leaving exile}
\chapter{Entering the land of Israel}
\chapter{Organization of the nation}
\chapter{Rewards of the nation}
\chapter{The giving of the Torah}
\chapter{The purpose of the Torah}
\chapter{The intention of the commandments}
\chapter{The land of Israel as the Holy Land}
\chapter{Holy Land for Holy Nation}
\chapter{The land of the covenant}
\chapter{Cherished nation, inheritance of the deer}
\chapter{Land of the deed}
\chapter{Land of wisdom}
\chapter{Breadth of opinion}
\chapter{A good land}
\chapter{Land of reward}
\chapter{The next world}
\chapter{The world of the soul}
\chapter{Tribulations in the grave}
\chapter{Resurrection of the dead}
\chapter{The righteous of the nations of the world}
\chapter{All Israel has a portion in the world to come}
\chapter{The time of the Messiah}
\chapter{The vision of the Messianic age}
\chapter{The final redemption}
\chapter{The purpose of eternal redemption}
\chapter{The footsteps of the Messiah}
\chapter{The ingathering of the exiles}
\chapter{Routes of return}
\chapter{Establishing our judges}
\chapter{Rebuilding Jerusalem}
\chapter{The light of Israel’s Messiah}
\chapter{The messenger of redemption}
\chapter{The influence of Israel’s Messiah}
\chapter{Perfecting the world under the kingdom of the Almightly}
\chapter{Resurrection of the dead}
\part{Perfect Faith}
\chapter{The righteous livs by his faith}
\chapter{Definition of faith}
\chapter{Achievign faith}
\chapter{Losing faith}
\chapter{The believer and the philosopher}
\chapter{Proofs of faith}
\chapter{The benefits of faith and the loss of heresy}
\chapter{Faigh and Torah}
\chapter{Torah and faith}
\chapter{Investigating faith and performing mitzvot}
\part{Free will}
\chapter{Free will}
\chapter{Determination and non-determination}
\chapter{Knowledge and memory}
\chapter{Origins of free will}
\chapter{Human beings and the world}
\chapter{Human beings and Torah}
\chapter{The nature of God’s Torah}
\chapter{The eternity of the Torah}
\chapter{Knowledge of God in human activities}
\chapter{Divine Providence}
\chapter{Rabbi Yehuda Halevi’s view}
\part{The Torah’s view on providence (as it appears in our opinion in the Torah, the Prophets, and Rabbinic writings)}
\chapter{Guiding the world}
\chapter{God cares for all his creations}
\chapter{God cares for the human race and for individuals}
\chapter{Suffering}
\chapter{Why do the wicked prosper}
\chapter{Bad things happen to good people and good things happen to bad people}
\part{7}
\chapter{Fear of God and knowledge of God}
\part{The believer and the freethinker in the revival of the nation}
\chapter{Love of the nation}
\chapter{Love of national literature}
\chapter{Love and respect for the nation’s spiritual achievements}
\chapter{Those with no portion in the world to come}
\chapter{Love of Zion and return to Zion}
\part{The believer and the freethinker in the State of Israel}
\chapter{The believer in the State of Israel}
\chapter{The freethinker in the State of Israel}
\chapter{The messianic age}
\part{Beginnings of knowledge of God}
\chapter{Fear of God}
\chapter{Seeking God}
\chapter{Spiritual awareness}
\part{Spirituality, unbelief, unawareness and alienation}
\chapter{Belief and disbelief}
\part{The nature of the soul}
\chapter{The nature of the soul}
\part{Saadia Gaon}
\end{document}
